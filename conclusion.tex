\chapter{Conclusion}
\epigraph{Il semble que la perfection soit atteinte, non quand il n'y a plus 
rien à ajouter mais quand il n'y a plus rien à retrancher}{--- \small{\textup{Saint-Exupéry,
Terre des hommes}}}

Ces 7 mois de stage en tant qu'assistant ingénieur chez VideoStitch m'ont beaucoup
plu et ont répondus à mes attentes.\\
Je souhaitais travailler dans une start-up et ce pour plusieurs raisons~: tout d'abord
je souhaitais avoir un aperçu de la chaîne de développement complète au sein d'une
équipe regroupant différentes compétences des métiers de l'ingénieur. Ma première
mission m'a permis de travailler sur le \textit{workflow} de l'équipe de développement
et sur l'architecture des logiciels de VideoStitch, et ainsi de comprendre cette chaîne
de bout en bout.\\
Je souhaitais également intégrer une entreprise innovant de nouveaux usages
numériques. Ma seconde mission m'a pleinement convaincu, en travaillant sur le 
système de \textit{plugins}. Passionné par le domaine de la photographie j'ai pu 
travailler avec de nombreuses caméras sur un produit d'imagerie numérique
et de vision par ordinateur.\\
\newline
La méthodologie Scrum fut très intéressante et avoir pu l'expérimenter me confirme
dans sa pertinence de gestion des projets logiciels. Je compte utiliser des méthodes
agiles pour la gestion de mes futurs projets .\\
Outre la méthodologie, la dynamique et la petite taille de l'équipe au sein de l'entreprise
a été très agréable et stimulante~: la communication est spontanée et très régulière,
utilisant les supports écrits comme oraux. C'est finalement l'équipe plus que les 
personnes qui apprend et progresse.\\
Quant au contenu du stage, sa richesse et la complexité du domaine de l'entreprise
m'a permis d'apprendre énormément sur le développement et l'architecture logiciel 
en C++ et en Qt. Outre les aspects techniques poussés et les nombreux concepts de
programmation et autres bonnes pratiques, je retiendrai particulièrement la citation 
inscrite au début de cette conclusion. Je la dois à Nicolas Lopez, suite à une soirée
relativement ardue de déboguage pour le compte d'un client~: \textit{Keep It Simple and Short}.\\
\newline
A l'issu de ce stage, j'ai donc décidé de suivre la filière \emph{Ingénierie
des Connaissances et des Supports d'Information}, pour faire suite à ce stage.\\
Enfin, j'espère avoir fourni à VideoStitch un travail à la hauteur des besoins
exprimés et c'est avec attention que je suivrai la sortie et l'évolution de Vahana VR.

