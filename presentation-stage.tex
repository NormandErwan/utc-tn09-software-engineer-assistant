\chapter{Présentation du stage}

\section{Le sujet}
Le titre du sujet du stage était initialement \enquote{\textit{Developer C++/Qt.
work on a challenging and immersive 360 video software}}.\\
Bien que très général, le domaine de l'entreprise se trouvait être très intéressant
et durant l'entretien qui s'est déroulé à la mi-mai, quelques sujets ont été
développés~: M. Burtey m'a donc proposé de travailler sur le Player d'une part,
à la suite d'Alexis Pontin et pour terminer son travail, et d'une autre part de
travailler au développement du nouveau produit VideoStitch Live, renommé par la
suite en Vahana VR, qui allait débuter dans quelques semaines.\\
M. Burtey n'était pas encore certain du lancement du développement de Vahana VR, 
et envisageait également à la conception du \textit{stitch} stéréoscopique 3D\footnote{Qui
consiste, brièvemenent, en la conception de deux équirectangulaires pour la même
scène filmée~: un pour chaque oeil en exploitant l'effet de la parallaxe, recréant 
un effet de relief\cite{videostitch-stereo}
\cite{image-stereoscopique}}, il me proposa également de travailler au développement
de la prise en charge de la 360 stéréo dans Videostitch Studio. Le sujet final serait
décidé la première semaine du stage.\\
\newline
Le stage fut finalement précédé par un CDD à temps partiel de 7 semaines, mes missions
se sont alors simplement étendues du début du CDD à la fin du stage, c'est-à-dire 
7 mois.\\
Le sujet réel a donc été arrêté à deux missions\footnote{La 3D s'est révélée être 
une tâche trop ardue, après un rapide état de l'art du sujet par les ingénieurs~: 
même si les articles de recherche et brevets sont nombreux, aucune application 
industrielle n'a encore aujourd'hui aboutis.}~:
\begin{enumerate}
\item L'intégration de VideoStitch Player et de Vahana VR au flux de développement et d'intégration
logiciel de VideoStitch, tout en assurant une maintenance sur le Player.
\item La conception, le développement et le déploiement d'un système
d'intégration de caméras et de cartes d'acquisitions dans Vahana VR.
\end{enumerate}
Ces deux missions trouvent comme sujet commun \emph{l'amélioration de \textit{workflow}
dans et pour les usages des ingénieurs de VideoStitch}. La première consiste en 
ce sens à l'amélioration du système et des outils de développement et d'intégration des logiciels
de l'entreprise; et la seconde permet par un système indépendant et modulaire de
nouvelles capacités à Vahana VR.

\section{Les méthodes de travail}

\section{Les outils utilisés}
