\chapter[Déplacer des fichiers entre des dépôts Git en préservant l'historique]
  {Déplacer des fichiers entre des dépôts Git en préservant l'historique\footnotemark}
\label{deplacer-historique-depots}

\footnotetext{Cette technique est inspirée d'une question sur le site stackoverflow dont voici
  le lien~: \url{http://stackoverflow.com/q/1365541}.}
  
L'objectif est de déplacer un dossier d'un dépôt A vers un dépôt B, tout en conservant
l'historique de ses modifications. Cependant le dépôt B peut déjà contenir des 
fichiers et un historique.
L'opération se déroule en deux étapes.

\paragraph{}
Premièrement il faut préparer le dossier du dépôt A, comme le montre le listing~
\ref{preparation-dossier}.
Après une création d'une nouvelle copie locale du dépôt A avec \mintinline{bash}{git clone} (ligne 1),
la commande \mintinline{bash}{git filter-branch} va supprimer tous les fichiers et
toutes les modifications qui ne concerne pas le dossier (ligne 2). Le dépôt contient
désormais à sa racine le contenu de ce dossier. Il faut le déplacer vers le nom du
dossier souhaité pour le dépôt B (ligne 3) puis enregistrer cette modification avec \mintinline{bash}{git commit} (ligne 4).
\begin{listing}
  \begin{minted}{bash}
    git clone <repository A url> && cd <repository A url>
    git filter-branch --subdirectory-filter <the directory> -- --all
    mkdir <the directory> && mv -k * <the directory>
    git add . && git commit
  \end{minted}
  \caption{Préparation du dossier du dépôt A}
  \label{preparation-dossier}
\end{listing}

\paragraph{}
Ensuite, il s'agit de déplacer le dossier tel que le montre le listing~\ref{deplacement-dossier}.
Après avoir créé une copie du dépôt B (ligne 1), la gestion décentralisée des dépôts
par Git permet de créer une \emph{connexion à la copie locale} du dépôt A (ligne 2) et d'en tirer fichiers
et historique (ligne 3). Le dossier a été déplacé, la copie du dépôt A peut être supprimée (ligne 4).
\begin{listing}
  \begin{minted}{bash}
    git clone <repository B url> && cd <repository B url>
    git remote add link-to-repo-A ../<repository A>
    git pull link-to-repo-A
    git remove rm link-to-repo-A && rm -r ../<repository A>
  \end{minted}
  \caption{Déplacement du dossier du dépôt A au dépôt B}
  \label{deplacement-dossier}
\end{listing}

