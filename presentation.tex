\chapter{Présentation de l'entreprise}
\section{VideoStitch}
\subsection{Présentation générale}
VideoStitch est une entreprise française d'édition de logiciels, qui conçoit et vend des logiciels de capture, de montage, de montage, de diffusion et de lecture de vidéos panoramiques à 360\°. Elle s'intéresse par extension au marché de la réalité virtuelle, ou encore appellée VR.
Son site internet est à l'adresse \url{http://www.video-stitch.com/}.
C'est une Société à Actions Simplifiées (SAS) basée sur Paris, qui emploie moins d'un dizaine de personnes actuellement. 
La plupart ayant été embauchés il y a quelques mois, c'est une société en forte croissance salariale. 
Nicolas Burtey en est le président.

\subsection{Historique}
VideoStitch a été fondée par ce même Nicolas Burtey en 2012.\\
Photographe diplômé de l'Ecole Lumière, s'étant fait connaître par la photo panoramique et la photo fish-eye, M. Burtey évolue par la suite dans le domaine de la vidéo panoramique. 
Il créé en 2011 sa société de production vidéo, Loop'In, avec laquelle il réalise une certain nombre de vidéos panoramiques pour de nombreux clients.
Un exemple notable est la série de publicités pour le constructeur automobile Renault : \url{}.\\
La vidéo panoramique, inconnue du grand public et étant un procédé très jeune, est lente et difficile à réaliser, beaucoup de tâches devant être faites de manière manuelle et répétitive, avec des outils inadaptés car non prévus à cette utilisation.
Si la création de photos panoramiques et panoramiques à 360\° est aisée, il manque toutefois un logiciel complet et mature permettant l'édition et le montage de vidéos panoramiques.\\
M. Burtey décide alors de créer VideoStitch et de concevoir cet outil. C'est une petite équipe de trois personnes qui se forme ainsi pour concevoir leur premier logiciel d'éditions et de montage de vidéos panoramiques : VideoStitch Studio.\\
Quand ce logiciel atteint une version stable, en 2014, VideoStitch oriente alors ses efforts vers un nouveau logiciel, Vahana VR, permettant cette fois-ci une réalisation et une diffusion en direct de vidéos panoramiques à 360\°.\\
\newline
Le but de la société est de permettre une réalisation aisée et de qualité de vidéos panoramiques à 360\°, de la capture des images à la diffusion de la vidéo finale, en passant par sa réalisation complète.


\section{Les marchés et les produits}
% La vidéo 360 panoramique et la VR, les principaux acteurs, bref historique et portrait actuel

\section{Produits et stratégies de l'entreprise}
% Studio, Vahana VR
% Player, réseaux sociaux, salons, rencontres entreprises

\section{Organisation de l'entreprise}
% Organisation de l'entreprise : les différentes personnes et articulation des rôles et des tâches.
% Organisation du travail : Agile/Scrum, JIRA, Confluence, Github et autres outils.

\section{Stratégies pour l'avenir et ambitions}
% Analyse de l'avenir des marchés, stratégies de positionnement et d'investissement, ambitions
