\chapter{Amélioration du flux de développement}
\epigraph{Il semble que la perfection soit atteinte, non quand il n'y a plus 
rien à ajouter mais quand il n'y a plus rien à retrancher}{--- \small{\textup{Saint-Exupéry,
Terre des hommes}}}

\section{Définition de la mission}
\subsection{Contexte}
Au début de ce stage, mi-juillet, la première tâche fut la prise de contact avec 
les méthodes de travail, les outils utilisés, les logiciels développés, et leurs
fluxs de développement. En effet, si jusqu'ici seul Studio était développé, Alexis
Pontin venait de terminer le projet du Player, et le développement de Vahana
VR commençait. Studio utilisait alors une architecture de développement particulière
et le Player tout comme Vahana VR commençaient leur développement selon leurs propres
\textit{workflows}.\\
\newline
Ce qui est appellé \textit{workflow} ou flux de développement est un terme pour désigner le 
\enquote{flux de travaux, [c'est-à-dire] une suite de tâches et
d'opérations effectuées par [...] un groupe de personnes}\cite{workflow}, ici l'équipe de développement
de VideoStitch~: c'est une représentation des différentes tâches à accomplir pour installer
son poste et pouvoir développer un des produits de l'entreprise, le compiler et le distribuer.
Ce n'est pas l'architecture des logiciels en eux même, ni leur développement,
mais comment, à partir des sources et des dépendances, il possible d'arriver au produit
compilé et distribuable au client.\\
Ce processus met en \oe uvre plusieurs concepts qui s'articulent autour du développement
logiciel proprement dit~:\cite{software-build}\cite{build-automation}
\begin{itemize}
  \item La gestion de versions, assurée ici avec Git\cite{gestion-versions}
  \item La chaîne de compilation, dépendante de l'OS cible et automatisée avec Buildbot \cite{chaine-compilation}\cite{integration-continue}
  \item Les tests d'intégration\cite{integration-continue}
  \item La création des installateurs et leur mise à disposition interne\\
\end{itemize}
%\newline
La mission porte essentiellement sur l'amélioration de la chaîne de compilation
pour les applications de VideoStitch, VideoStitch Studio, Vahana VR et VideoStitch
Player, et la création des installateurs. Ce travail a été très complémentaire avec
celui de Jean Duthon, qui a amélioré le flux de développement de VideoStitch SDK,
la gestion des versions et les tests d'intégration.

\subsection{Le flux de développement de VideoStitch Studio} 
Le modèle 

\subsection{Problématique}
Représenter ce flux présente l'intérêt de pouvoir ensuite le simplifier et d'automatiser
un maximum des processus requis. Et cela était devenu nécessaire, pour deux raisons~:
\begin{enumerate}
  \item L'entreprise commençait sa croissance, impliquant de former les nouveaux arrivants
  à un flux de développement devenue complexe à mesure du temps. Il fallait en
  simplifier le fonctionnement maintenant que Studio et son développement était
  devenus matures\footnote{La première version était sortie, et la beta de la seconde
  version était en cours}, ce qui permettrait de gagner en efficacité sur le développement
  proprement dit.
  \item Vahana VR et le Player avaient débuté leurs développements; pourtant ces trois
  produits, avec Studio, ont en commun l'utilisation de VideoStitch SDK~: les logiciels
  étant très proches, il était nécessaire d'unifier leurs fluxs de développement.
\end{enumerate}

\subsection{Objectifs}
Les objectifs retenus ont été~:
\begin{itemize}
  \item Faire évoluer le flux de développpement vers une gestion multi-logicielle et 
  et en simplifier le fonctionnement.
  \item Intégrer VideoStitch Player et Vahana VR au flux de développement.
  \item A cela s'est ajouté, avec le départ d'Alexis, un suivi du Player pour en assurer
  sa maintenance.
\end{itemize}


\section{Réalisation}
\subsection{Intégration des dépendances du VideoStitch Player}

\subsection{Evolution de l'architecture du dépôt VideoStitch-apps}

\subsection{Intégration du VideoStitch Player et de Vahana VR au dépôt VideoStitch-apps}

\subsection{Automatisation de la chaîne de compilation}

\subsection{Génération des installateurs}

\subsection{Maintenance du VideoStitch Player}
