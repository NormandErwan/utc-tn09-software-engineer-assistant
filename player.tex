\chapter{Amélioration du workflow de développement}
\epigraph{Il semble que la perfection soit atteinte, non quand il n'y a plus 
rien à ajouter mais quand il n'y a plus rien à retrancher}{--- \small{\textup{Saint-Exupéry,
Terre des hommes}}}

\section{Définition de la mission}
\subsection{Contexte}
Au début de ce stage, mi-juillet, la première tâche fut la prise de contact avec 
les méthodes de travail, les outils utilisés, les logiciels développés, et leurs
\textit{workflows} de développement. En effet, si jusqu'ici seul Studio était développé, Alexis
Pontin venait de terminer le projet du Player, et le développement de Vahana
VR commençait. Studio utilisait alors une architecture de développement particulière
et le Player tout comme Vahana VR commençaient leur développement selon leurs propres
\textit{workflows}.\\
\newline
Ce qui est appelé \textit{workflow} dans ce rapport est un terme pour désigner le 
\enquote{flux de travaux, [c'est-à-dire] une suite de tâches et
d'opérations effectuées par [...] un groupe de personnes}\cite{workflow}, ici l'équipe de développement
de VideoStitch~: c'est une représentation des différentes tâches à accomplir pour, depuis un poste de travail vierge, 
l'installer et pouvoir travailler un des produits de l'entreprise, le compiler et le distribuer.
Ce n'est pas l'architecture des logiciels en eux même, ni leur développement,
mais comment, à partir des sources et des dépendances, il possible d'arriver au produit
compilé et distribuable au client.\\
Ce processus met en \oe uvre plusieurs concepts qui s'articulent autour du développement
logiciel proprement dit~:\cite{software-build}\cite{build-automation}
\begin{itemize}
  \item La gestion de versions, assurée ici avec Git\cite{gestion-versions}, qui permet
  de récupérer les codes sources.
  \item La chaîne de compilation, dépendante de l'OS cible et automatisée avec Buildbot 
  \cite{chaine-compilation}\cite{integration-continue}.
  \item Les tests d'intégration\cite{integration-continue} réalisés par Buildbot.
  \item La création des installateurs et leur mise à disposition sur une page de téléchargement interne.
\end{itemize}
\ \newline
Cette mission porte essentiellement sur l'amélioration de la chaîne de compilation
pour les applications de VideoStitch, VideoStitch Studio, Vahana VR et VideoStitch
Player, et la création des installateurs. Ce travail a été très complémentaire avec
celui de Jean Duthon, qui a amélioré le \textit{workflow} de développement de VideoStitch SDK,
la gestion des versions et les tests d'intégration.

\subsection{Le \textit{workflow} de développement de VideoStitch Studio}
\label{workflow-studio}
Les codes sources et les dépendances avaient été réparties en quatre dépôts Git~: 
\begin{itemize}
  \item VideoStitch-lib, contenant le VideoStitch SDK
  \item VideoStitch-base, contenant des code sources partagés avec le dépôt du VideoStitch Player
  \item VideoStitch-apps, contenant VideoStitch Studio
  \item VideoStitch-deps, contenant les dépendances Windows de VideoStitch Studio 
  nécessaires à sa compilation\footnote{Présentées plus en détails dans \cf{integration-dependances-player}}
\end{itemize}
Sous les systèmes Linux et OS X, il a été décidé d'utiliser les gestionnaires
de paquets apt et Macport pour installer les dépendances logicielles de Studio. Pour
Windows, le dépôt VideoStitch-deps tenait les dépendances sous des versions stables et testées.\\
\newline
Compiler et créer l'installateur de Studio depuis un poste de travail vierge requérait alors
certaines étapes, représentées sous forme d'un diagramme de séquence sur la figure~\ref{workflow-studio}.
Ces étapes étaient essentiellement des appels à des commandes et à des scripts internes
(écrits en bash pour Linux / Mac et en batch pour Windows) pour faire le lien entre les trois dépôts~:
\begin{enumerate}
  \item Cloner les dépôts VideoStitch-deps, VideoStitch-base et VideoStitch-apps.
  \item Générer une version du logiciel via un script présent sur VideoStitch-apps.
  \item Récupérer la dernière version du VideoStitch SDK compilée par Buildbot, 
  via un script présent sur VideoStitch-apps.
  \item Copier, via un script présent sur VideoStitch-base, les codes sources partagés.
  \item Copier, via un script batch présent sur VideoStitch-deps, les dépendances pour Windows.
  \item Générer le Makefile avec le logiciel qmake, fournis avec Qt.
  \item Compiler avec le logiciel jom, fournis avec Qt.
  \item Générer l'installateur avec un script présent sur VideoStitch-apps.
\end{enumerate}
\begin{figure}
  \centering
  \caption{Diagramme de séquence du \textit{workflow} de Studio (juillet 2014)}
	\label{workflow-studio}
\end{figure}

\subsection{Problématique}
Représenter ce flux présente l'intérêt de pouvoir ensuite le simplifier et d'automatiser
un maximum des processus requis. Et cela était devenu nécessaire, pour deux raisons~:
\begin{itemize}
  \item L'entreprise commençait sa croissance, impliquant de former les nouveaux arrivants
  à un \textit{workflow} de développement devenue complexe à mesure du temps. Il fallait en
  simplifier le fonctionnement maintenant que Studio et son développement était
  devenus matures\footnote{La première version était sortie, et la bêta de la seconde
  version était en cours}, ce qui permettrait de gagner en efficacité sur le développement
  proprement dit.
  \item Vahana VR et le Player avaient débuté leurs développements; pourtant ces trois
  produits, avec Studio, ont en commun l'utilisation de VideoStitch SDK~: les logiciels
  étant très proches, il était nécessaire d'unifier leurs \textit{worklows} de développement.
\end{itemize}

\subsection{Objectifs}
Les objectifs retenus ont été~:
\begin{itemize}
  \item Faire évoluer le \textit{workflow} de développement vers une gestion multi-logicielle sur le dépôt VideoStitch-apps 
  et en simplifier le fonctionnement.
  \item Intégrer VideoStitch Player et Vahana VR à ce \textit{workflow} de développement.
  \item A cela s'est ajouté, avec le départ d'Alexis, un suivi du Player pour en assurer
  sa maintenance.
\end{itemize}


\section[Réalisation]{Réalisation
\protect\footnote{Nicolas Lopez, Julien Fond et Jean Duthon m'ont beaucoup aidé et, par 
leurs corrections, ont largement contribué aux résultats présentés ici.}}

\subsection{Intégration des dépendances du VideoStitch Player}
\label{integration-dependances-player}
Un programme dépend, pour sa compilation et son exécution, d'autres \emph{paquets} logiciels
\cite{dependance-logicielle}. Ce sont des archives contenant des \emph{bibliothèques logicielles},
c'est-à-dire des ensembles de fonctions déjà compilés et utilisable par le programme.\\
Une bibliothèque logicielle sous Windows se constitue de trois types de fichier\cite{bibliotheque-logicielle}~:
\begin{itemize}
  \item Les \textit{headers} (fichiers .h), qui déclarent les prototypes des fonctions utilisables
  et accessibles de la bibliothèque.
  \item Les bibliothèques (fichiers .lib et .dll), qui contiennent le code des fonctions.
  Les fichiers .lib seront copiés dans le programme, alors que les fichiers .dll seront
  chargés lors du démarrage du programme\cite{bibliotheque-logicielle}.
\end{itemize}
\ \\
En plus d'être spéficique à un système d'exploitation, un programme, et donc
ses dépendances, est spécifique à la plateforme visée (32 bits ou 64 bits)\cite{64-bit-computing}
mais aussi à la configuration compilée (\textit{debug} ou \textit{release})\cite{msdn-debug-release}.\\
La configuration \textit{debug} permet aux développeur de générer une version pour le débogage,
quand la configuration \textit{release} permet de générer une version finale et optimisée destinée
à l'usage du programme. \\
Enfin, le choix de la plateforme dépend du processeur et du
système d'exploitation du client. Les produits de VideoStitch sont distribués
seulement en 64 bits~: les machines pour les faire fonctionner demandent du matériel
récent donc en 64 bits, et cela réduit la maintenante à une seule plateforme.\\
\newline
Le dépôt VideoStitch-deps était déjà logiquement organisé sous la forme présentée figure~\ref{dirtrees-deps}.
\begin{figure}
\dirtree{%
 .1 /.
 .2 src\DTcomment{Archives des versions des dépendances utilisées}.
 .2 x64\DTcomment{La plateforme 64 bits Windows}.
 .3 bin\DTcomment{Les bibliothèques .dll}.
 .4 debug\DTcomment{Les .dll spécifiques \textit{debug}}.
 .4 release.
 .3 include\DTcomment{Les fichiers \textit{headers} .h}.
 .3 lib\DTcomment{Les bibliothèques .lib}.
 }
\caption{Dépôt VideoStitch-deps}
\label{dirtree-deps}
\end{figure}
Cette organisation convenait aux nouveaux besoins et a été gardée ainsi. Une séparation
des dépendances par logiciel aurait été possible, mais complexifiant inutilement 
la structure. De plus, les dépendances en commun entre les trois logiciels
sont utilisées dans les même versions. \\
\newline
Ainsi, dans le cadre du Player, deux dépendances ont été ajoutées~:
\begin{itemize}
  \item \textbf{libVLC}~: SDK de VLC, cette bibliothèque fournit des capacités multimédias.\cite{libvlc}
  \item \textbf{Oculus SDK}~: cette bibliothèque permet à une application d'exploiter le 
  casque de réalité virtuelle Oculus Rift.\cite{oculus-developer-guide}
\end{itemize}
Les dépendances Qt et crashrpt\footnote{Envoi de rapports automatiques lors d'un
\textit{crash} du logiciel \url{https://code.google.com/p/crashrpt/}} étaient déjà 
présentes et utilisées par Studio.\\
En parallèle, les dépendances libVLC et Oculus SDK ont été installées sur les Buidlbots Mac et Linux.\\
Quand aux dépendances utilisées par Vahana VR, elles étaient les même que celles 
utilisées Studio, donc déjà présentes dans le dépôt.\\
\newline
Enfin, le script utilisé pour copier ces dépendances vers VideoStitch-apps a été
revu et amélioré, à une version très simple~:
code

\subsection{Intégration au \textit{workflow} du VideoStitch Player et de Vahana VR}
La première étape a été de rapatrier le dépôt VideoStitch-base et de l'inclure
avec son historique dans le dépôt VideoStitch-apps. L'annexe \ref{deplacer-historique-depots}
présente plus en détail l'opération permettant, grâce à Git, de déplacer un ensemble 
de fichiers d'un dépôt à un autre tout en préservant l'historique des modifications.\\
De la même manière, furent importés les codes sources de Player et Vahana VR de 
leurs dépôts respectifs.\\
L'opération fut relativement aisée, car le dépôt VideoStitch-apps présentait déjà
une organisation regroupant plusieurs programmes, mais centrés sur le seul logiciel
VideoStitch Studio\footnote{On distingue logiciel et programme, dans le sens où un logiciel
est composé de un ou plusieurs programmes, c'est-à-dire des fichiers binaires, et de
fichiers de configurations, images, etc.}. L'architecture ne nécessitait alors que quelques modifications, et
avait été fixée à la hiérarchie présentée figure~\ref{dirtree-apps}.
\begin{figure}
\dirtree{%
  .1 /.
  .2 bin\DTcomment{Dossier de compilation et dll copiées de VideoStitch(deps)}.
  .3 x64.
  .4 debug.
  .4 release.
  .2 installer\DTcomment{Scripts de conception des installateurs}.
  .2 external\_deps\DTcomment{Headers et lib copiés de VideoStitch-deps (Windows)}.
  .3 include.
  .3 lib.
  .2 packager\DTcomment{Scripts de conception des installateurs}.
  .2 src\DTcomment{Codes sources des applications}.
  .3 libvideostitch\DTcomment{Contient les headers copiés du VideoStitch SDK}.
  .3 libvideostitch-base\DTcomment{Programme commun au Player, Studio et Vahana VR}.
  .3 videostitch-player-gui\DTcomment{VideoStitch Player}.
  .3 libvideostitch-gui\DTcomment{Programme commun à Studio et Vahana VR seulement}.
  .3 videostitch-studio-gui\DTcomment{VideoStitch Studio}.
  .3 videostitch-cmd.\DTcomment{Programme inclus avec Studio}.
  .3 vahana-vr\DTcomment{Vahana VR}.
}
\caption{Dépôt VideoStitch-apps}
\label{dirtree-apps}
\end{figure}
Le dossier de compilation est fixé à bin/x64 pour tous les systèmes~: Windows, Linux 
et Mac OS X.
\ \\
L'architecture convenant, les fichiers projets Qt nécessitaient d'être refactorisés,
c'est-à-dire les réécrires pour les adapter aux nouveaux besoins voulu. Ici, les
trois applications étant proches, il est intéressant de partager un maximum de leurs
codes sources sous des programmes communs d'une part et d'autre part de partager
la configuration des différents projets.\\
Les applications son

\ \\
Enfin, quelques scripts ont été réécrits. Comme vu avec \cf{workflow-studio}, le
développeur doit récuperer la bibliothèque du VideoStitch SDK, et les dépendances,
pour Windows. Ces ressources se trouvent sur


\subsection{Génération des installateurs}
Si la compilation des trois produits est désormais possible, il s'agit désormais
de pouvoir distribuer au client une version finale et utilisable.\\
Le logiciel doit être fournis en version \textit{release} et optimisée, ainsi que ses dépendances associées.
Cependant, comme il a été vu à la section précédente, la compilation produit un
certain nombre de fichiers supplémentaires au programme compilé. Il s'agit donc
de construire un paquet contenant seulement les fichiers nécessaires, le logiciel
et ses dépendances.

\subsubsection{Windows}
Sudio et Player utilisaient tout deux le logiciel InnoSetup, qui permet de créer des
installateurs, les paquets Windows. 
win tout inclure inno setup cherry pick parmis le mess de bin/

\subsubsection{Mac}

\subsubsection{Linux}


\subsection{Automatisation de la chaîne de compilation}
La génération des installateurs Windows et des paquets Linux / Mac est cependant
toujours fastidieuse et gagnerait à être automatisée. L'équipe se concentre avant
tout au développement, quand l'intégration continue est réalisée par Buildbot. Un
serveur va donc, à chaque poussée de code sur les dépôts Git, générer le VideoStitch
SDK et les trois produits, ainsi que leurs installateurs. Ces derniers sont destinés
à un usage interne de test, et, lors des \textit{release} des produits, destinés
aux clients.


iste les étapes et les écrits sur le buildbot en python 

\subsection{Maintenance du VideoStitch Player}
Enfin, quelques apports mineurs et rapides ont été apportés, pour améliorer l'usage
du Player lors des présentations dans les salons et conférences.
\subsubsection{Support de l'Oculus Rift DK2}

\subsubsection{Passer le \textit{Health and Safety Warning} de l'Oculus}

\subsubsection{Ajout d'options en ligne de commande}


\section{Bilan et suite}
Cette première mission occupa les deux premiers mois du stage, et permis d'acquérir
le fonctionnement du \textit{workflow} de développement de l'entreprise, et de l'améliorer.
Les modifications apportées, même si elles ont été parfois mineures à un système qui 
fonctionnait déjà correctement pour les usages précédents, ont cependant permis dans leur globalité~:
\begin{itemize}
 \item De compiler depuis un même dépôt, quelque que soit l'OS de la machine, les trois applications de VideoStitch.
 \item D'automatiser la compilations et la génération des paquets de ces trois applications 
 sur Buidlbot, rendant les dernières versions immédiatement utilisables par n'importe 
 quelle personne de VideoStitch.
 \item D'apporter les modifications nécessaires à l'usage du Player.
 \item De permettre l'installation de nouveau postes de travail ainsi que le workflow 
 de développement aisé à l'utilisation, comme le montre la figure~\ref{workflow-final}.
\end{itemize}
\begin{figure}
  \centering
  \caption{Diagramme de séquence du \textit{workflow} de développement des applications de VideoStitch (Player, Studio, Vahana VR) (septembre 2014)}
	\label{workflow-final}
\end{figure}
\ \\
Ce système pourrait encore être amélioré, ou même à être repris de zéro. Cependant, 
une telle réécrire demanderait nécessairement beaucoup plus de temps pour
produire un système certes plus correct au regard des besoins actuels, mais moins mature
et donc plus sujet à des cas d'utilisations non prévus ou problématiques. De plus,
il aurait fallu changer les habitudes de l'équipe de développement.\\
Cependant, les cas d'utilisations non prévus sont apparus lors de cette mission,
mais sont rapidement corrigés par l'équipe. Bien souvent, cela est dû à des différences
d'interprétations du bash entre Linux et Mac OS X, ou par une différence de code avec
le script batch correspondant. Les scripts pourraient être écrits en Python, pour
éviter ces différences et ces erreurs. Leur écriture et leur maintenance serait 
en outre facile, Python étant relativement simple dans son approche et ses usages.\\
Enfin, VideoStitch Player nécessiterait d'être amélioré~: la vidéo
4K n'est pas correctement lue, libVLC ne semblant pas supporter correctement cette
haute résolution, et le logiciel gagnerait à être finis autant du points de vue
du code, de l'optimisation générale que de l'interface. Ce travail nécessiterait 
d'être consacré par quelques sprints de l'équipe.
