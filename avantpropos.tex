\chapter*{Avant-propos}
\addcontentsline{toc}{chapter}{Avant-propos}

\section*{Remerciements}
\addcontentsline{toc}{section}{Remerciements}
Je souhaite remercier tout d'abord l'UTC et VideoStitch pour m'avoir permis de 
réaliser ce stage technique d'assistant ingénieur. Ces 7 mois très riches m'ont 
permis de constituer une première expérience technique et professionnelle solide 
en ingénierie logicielle. Ce stage fut une transition pour moi de la première 
année commune du Génie Informatique de l'UTC à la filière de spécialisation 
d'Ingénierie des Connaisances et des Systèmes d'Informations.\\
\newline
Je remercie sincèrement Nicolas Burtey, pour m'avoir accueilli dans son entreprise,
pour la confiance qu'il m'a témoignée en acceptant ma candidature ainsi que pour
son suivi régulier de mon stage et du temps qu'il m'a consacré à mes nombreuses
questions sur l'entrepreneuriat.\\
Je remercie en particulier Andrés Peralta, Nicolas Lopez et Julien Fond avec qui 
j'ai le plus souvent travaillé, pour leurs aides et leurs nombreux conseils.\\
Je remercie au même titre toute l'équipe, ingénieurs, commerciaux, et stagiaires 
de VideoStitc avec qui j'ai pu longuement travailler :
Henri Rebecq, Nils Duval, Rodolphe Fouquet, Wieland Morgenstern, Raphaël 
Lemoine, Jean Vittor, Aksel Piran, Liesbeth De Mey, Florent Melchior, Rodolphe 
Chartier, Alexis Pontin ainsi que les deux UTCéens avec qui j'ai effectué mon 
stage : Marie Chatelin et Jean Duthon. Merci à tous pour leurs aides, 
les nombreux partages et connaissances échangés. Venir travailler fut un plaisir 
et une motivation tous les jours grâce à eux tous et à l'excellente ambiance dans l'équipe.\\
\newline
Enfin, je remercie toutes les personnes qui m'ont consacré de leur temps au support 
de leurs produits, et merci à ceux qui les ont conçus. Notre travail est entouré 
d'outils qui innovent sans cesse et où nous prenons notre place. Il est bon de 
se rappeller que toute innovation a ses fondations; et j'espère que les briques 
que j'ai apportées seront la source de nouvelles idées.

\newpage
\section*{Résumé}
\addcontentsline{toc}{section}{Résumé}
J'ai réalisé un stage de 7 mois dans la start-up VideoStitch, basée sur Paris et qui
conçoit des logiciels de vidéo panoramique à 360\degree. Au sein d'une équipe utilisant
la méthode Scrum, j'ai réalisé deux missions d'amélioration et de conception de
\textit{workflows} à l'usage de cette équipe~: d'une part j'ai simplifié et fait évoluer
le \textit{workflow} de développement vers une gestion multi-logicielle
tout en intégrant deux nouveaux produits; d'autre part, j'ai développé
et déployé un système de \textit{plugins} augmentant les capacités entrées/sorties
des logiciels de l'entreprise. Sur ces deux projets, il s'agissait de trouver le
juste milieu entre réaliser le \textit{workflow} le plus complet et simple, 
quitte à factoriser des solutions déjà existantes, et réaliser \textit{workflow} 
le plus rapide à déployer, permettant de le développement en tant que tel des produits. 
À l'échelle de la start-up il s'agissait de positionner le curseur entre qualité
et délai pour tenir les coûts.

\subsection*{Mots-clés}
\addcontentsline{toc}{section}{Mots-clés}
Génie Logiciel, Programmation Orientée Composants, \textit{plugin}
Programmation Orientée Objet, Git 
