\chapter*{Avant-propos}
\addcontentsline{toc}{chapter}{Avant-propos}

\section*{Remerciements}
\addcontentsline{toc}{section}{Remerciements}
Je souhaiterais remercier tout d'abord l'UTC et VideoStitch pour m'avoir permis de réaliser ce stage technique d'assistant ingénieur. Ces 7 mois très riches m'ont permis de constituer une première expérience technique et professionnelle solide en ingénierie logicielle. Ce stage fut une transition pour moi de la première année commune du Génie Informatique de l'UTC à la filière de spécialisation d'Ingénierie des Connaisances et des Systèmes d'Informations.
\\
\newline
Je remercie sincèrement Nicolas Burtey, pour m'avoir accueilli dans son entreprise et pour la confiance qu'il m'a témoignée en acceptant ma candidature.\\
Et je remercie toute l'équipe, ingénieurs, commerciaux, et stagiaires de VideoStitch avec qui j'ai pu longuement travailler : Andrés Peralta, Nicolas Lopez, Julien Fond, Henri Rebecq, Nils Duval, Rodolphe Fouquet, Wieland Morgenstern, Raphaël Lemoine, Jean Vittor, Aksel Piran, Liesbeth De Mey, Florent Melchior, Rodolphe Chartier, Alexis Pontin ainsi que les deux UTCéens avec qui j'ai effectué mon stage : Marie Chatelin et Jean Duthon. Merci pour leurs aides, les nombreux partages et connaissances échangés. Venir travailler fut un plaisir et une motivation tous les jours grâce à eux tous et à l'excellente ambiance dans l'équipe.
\\
\newline
Je remercie l'incubateur Télécom Paris Tech de Paris XIV où s'est déroulé le stage.
\\
\newline
Enfin, je remercie toutes les personnes qui m'ont consacré de leur temps au support de leurs produits, et merci à ceux qui les ont conçus. Notre travail est entouré d'outils qui innovent sans cesse et où nous prenons notre place. Il est bon de se rappeller que toute innovation a ses fondations; et j'espère que les briques que j'ai apportées seront la source de nouvelles idées.


\section*{Résumé}
\addcontentsline{toc}{section}{Résumé}
\begin{abstract}
...
\end{abstract}
