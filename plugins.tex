\chapter{Développement d'un système de plugins d'entrées/sorties}

\section{Définition de la mission}
\subsection{Contexte}
Avec le début du développement de Vahana VR, la question des entrées et des sorties
du logicel s'était posée. En effet, Studio étant un logiciel de montage de vidéos
360, ses seules entrées possibles sont les fichiers vidéos issues des caméras de la monture,
cf. \cf{importation-videos}, et la sortie est l'équirectangulaire représentant
la vidéo 360 finale, cf. \cf{exportation}.\\
Ce qui est appellé \emph{entrée} ici sont les données envoyées au programme, et
les \emph{sorties} sont les données émises par ce programme en retour.\\
Pour permettre la vidéo 360 en direct, Vahana VR s'affranchit des fichiers
vidéos importés et propose de capturer directement les images des caméras, via
des cartes d'acquisitions branchées sur la machine. De même le logiciel peut émettre 
plusieurs fluxs en sorties permettant par exemple le \textit{streaming}, ou encore la réémissions vers
une carte d'acquisition, en plus de l'export d'un equirectangulaire sous forme de fichier
sur le disque dur.\\
\begin{figure}
  \centering
  \caption{Schéma des entrées/sorties de Vahana VR}
\end{figure}

\subsection{Problématique}
Dès lors, Vahana VR étant destiné à des sociétés de production, le logiciel doit
être compatible avec un maximum de standards, caméras et cartes d'acquisitions vidéos
pour être facilement adopté. Cependant, tout matériel informatique récquérant des pilotes
\footnote{Programme permettant au système d'exploitation d'interagir avec ce matériel\cite{pilote-informatique}.}
, et, les besoins des clients en entrées-sorties n'étant pas les mêmes, il n'était
alors pas possible de concevoir un logiciel monolithique contenant tous les programmes 
d'entrées-sorties supportés~: chaque installation du logiciel aurait requis au client 
une installation de l'ensemble des pilotes.\\
De plus, un client pourrait souhaiter utiliser du matériel entrée/sortie encore
non supporté. Il serait intéressant qu'il puisse réaliser son propre développement
dans l'ecosystème de Vahana VR; ainsi le logiciel pourrait devenir virtuellement
compatible avec n'importe quel standard, caméra ou carte d'acquisition.\\
Il fallait donc développer un système assurant la compatibilité
entre Vahana VR et ces matériels, et intégrant les contraintes exposées.\\

\subsection{Objectifs}
Au début de ce stage, une solution avait déjà été esquissée et était déjà en partie
mise à l'\oe uvre. Cependant un certain travail était encore nécessaire.\\
L'approche retenue fut celle de la \emph{Programmation Orientée Composant} (POC),
qui permet une certaine modularité dans l'architecture du projet\cite{poc}. Chaque nouvelle
carte d'acquisition donnera lieu à un nouveau composant, sous forme d'un \textit{plugin},
qui pourra être distribué, chargé et utilisé si nécessaire\cite{plugin}.\\
\newline
Les objectifs retenus ont donc été~:
\begin{itemize}
  \item Développer les entrées HDMI\footnote{\textit{High Definition Multimedia Interface}, 
  norme de diffusion audio/vidéo numériquei\cite{hdmi}.} et SDI\footnote{\textit{Serial Digital Interface}, 
  protocole de diffusion vidéo numérique\cite{sdi}.}.
  \item Développer les sorties HDMI et SDI.
  \item Développer des entrées-sorties Ethernet et PCIe\footnote{\textit{PCI Express}, standard
  de connexion de cartes d'extension sur la carte mère d'un ordinateur\cite{pci-express}.}
  spécifiques à la demande d'un client.
  \item Déployer l'ensemble des \textit{plugins} sur un dépôt séparé.
  \item Documenter et vérifier la bonne intégration des plugins dans Vahana VR.
\end{itemize}

\begin{figure}
  \centering
  \begin{minipage}[t]{0.2\textwidth}
    \centering
    %\includegraphics[width=3cm]{images/}
    \caption{HDMI}
  \end{minipage}%
  \hspace{0.03\textwidth}
  \begin{minipage}[t]{0.2\textwidth}
    \centering
    %\includegraphics[width=3cm]{images/}
    \caption{SDI}
  \end{minipage}%
  \hspace{0.03\textwidth}
  \begin{minipage}[t]{0.2\textwidth}
    \centering
    %\includegraphics[width=3cm]{images/}
    \caption{Ethernet}
  \end{minipage}%
  \hspace{0.03\textwidth}
  \begin{minipage}[t]{0.2\textwidth}
    \centering
    %\includegraphics[width=3cm]{images/}
    \caption{PCIe}
  \end{minipage}
  \caption{Illustration des différents connecteurs utilisées par Vahana VR}
\end{figure}

\section{Réalisation}
\subsection{Architecture de la solution}

\subsection{Conception type d'un plugin}

\subsection{Quelques difficultés et solutions spécifiques}


\section{Déploiement}
\subsection{Création du dépôt VideoStitch-IO}

\subsection{Documentation}

\subsection{Tests et \textit{QA} de l'intégration des plugins}


\section{Bilan et suite}

